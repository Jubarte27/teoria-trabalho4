\documentclass{article}
\usepackage{mainpreamble}

\title{\textbf{Redução de problemas \vskip 1em \smaller{Teoria da Computação N}}}

\author{\ldots}
\author{Jorgefran Souza Batista (00172589)}
\author{Matheus Machado Cezar (00597894)}
\author{Richard Muniz Rosa (00327098)}
\affil{Universidade Federal do Rio Grande do Sul}
\date{09/12/2025}

\begin{document}

\baselineskip=12pt
\maketitle

\newpage
\tableofcontents
\newpage

\section{Problema da Aceitação da Palavra Vazia}
Prove que \PAPV\ é um problema indecidível utilizando uma redução envolvendo \PP.
\subsection{Problemas}

\subsubsection{Problema da Parada (\PP)}
\begin{itemize}
    \item Entrada: um par $(M, w)$, onde $M$ é uma máquina de Turing sobre o alfabeto $\Sigma$ e $w \in\Sigma^\star$.
    \item Pergunta: $w\in\ (\ACEITA{M}\cup\REJEITA{M})$?
\end{itemize}

\subsubsection{Problema da Aceitação da Palavra Vazia (\PAPV)}
\begin{itemize}
    \item Entrada: uma máquina de Turing $M$ sobre alfabeto $\Sigma$.
    \item Pergunta: $\epsilon\in\ACEITA{M}$?
\end{itemize}

\subfile{q1/q1}

\section{Problema da Rejeição em Número Ímpar de Passos}
\PRNIP\ é decidível ou indecidível? Prove a sua resposta.
\subsection{Problemas}

\subsubsection{Problema da Rejeição em Número Ímpar de Passos (\PRNIP)}
\begin{itemize}
    \item Entrada: um par $(M, w)$, sendo M uma Máquina de Turing sobre alfabeto $\Sigma$, e $w\in\Sigma^\star$ uma palavra de entrada para $M$
    \item Pergunta: $w\in\REJEITA{M}$ tendo a computação levado um número ímpar depassos para chegar na rejeição?
\end{itemize}

\subfile{q2/q2}

\end{document}


