\documentclass{article}
\usepackage{mainpreamble}

\title{\textbf{Redução de problemas \vskip 1em \smaller{Teoria da Computação N}}}

\author{\ldots}
\author{Jorgefran Souza Batista (00172589)}
\author{Matheus Machado Cezar (00597894)}
\author{Richard Muniz Rosa (00327098)}
\affil{Universidade Federal do Rio Grande do Sul}
\date{09/12/2025}

\begin{document}

\baselineskip=12pt
\maketitle

\newpage
\tableofcontents
\newpage

\newcommand{\func}[2]{#1(#2)}
\newcommand{\form}[1]{\textcolor{blue}{\ensuremath{#1}}}

\section{Problema da Aceitação da Palavra Vazia}
\begin{itemize}
    \begin{item}
        Problema da Parada (PP). \\
        Entrada: um par \form{(M, w)}, onde \form{M} é uma máquina de Turing sobre o alfabeto \form{\Sigma} e \form{w \in\Sigma^\star}. \\
        Pergunta: \form{w\in\ (\func{ACEITA}{M}\cup\func{REJEITA}{M})}?
    \end{item}
    \begin{item}
        Problema da Aceitação da Palavra Vazia (PAPV). \\
        Entrada: uma máquina de Turing \form{M} sobre alfabeto \form{\Sigma}. \\
        Pergunta: \form{\epsilon\in\func{ACEITA}{M}}?
    \end{item}
\end{itemize}

Prove que PAPV é um problema indecidível utilizando uma redução envolvendo PP\@.
\subfile{q1/q1}

\section{Problema da Rejeição em Número Ímpar de Passos}

\begin{itemize}
    \begin{item}
        Problema da Rejeição em Número Ímpar de Passos (PRNIP). \\
        Entrada: um par \form{(M, w)}, sendo M uma Máquina de Turing sobre alfabeto \form{\Sigma}, e \form{w\in\Sigma^\star} uma palavra de entrada para M \\
        Pergunta: \form{w\in\func{REJEITA}{M}} tendo a computação levado um número ímpar depassos para chegar na rejeição?
    \end{item}
\end{itemize}

PRNIP é decidível ou indecidível? Prove a sua resposta.
\subfile{q2/q2}

\end{document}


