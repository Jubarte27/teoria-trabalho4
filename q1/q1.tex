\documentclass[../main.tex]{subfiles}
\usepackage{mainpreamble}

\begin{document}
\subsection{Demonstração}

Vamos construir uma redução $r : \PP\Rightarrow\PAPV$. Considere o seguinte algoritmo $r$ que, ao receber de entrada uma instância $(M, w)$ de \PP\, retorna uma instância $\func{r}{M, w} = M'$ de \PAPV\, onde $M'$ é a máquina de Turing que, para toda palavra $t$ de entrada, executa os seguintes passos:

\begin{enumerate}
    \item Apague $t$ da fita, volte o cursor para o início, escreva $w$ na fita e volte o cursor para o início. Vá para o passo 2
    \item Simule $M$ com entrada $w$. Se a simulação parar (aceitando ou rejeitando), aceite.
\end{enumerate}

Resta agora provarmos que a redução r descrita acima está correta:

\begin{itemize}
\item Suponha que (M, w) ∈ Y(PARADA), então M', por definição, ao iniciar seu processamento com a fita vazia, escreve w na fita, simula M com entrada w e eventualmente para. Logo r(M, w) = M' ∈ Y(PARADA-VAZIA)
\item Suponha que (M,w) \in N(PARADA), então M', por definição, entra em loop infinito ao iniciar seu processamento com a fita vazia. Logo r(M, w) = M′ ∈ N(PARADA-VAZIA)
\end{itemize}

\end{document}